\documentclass{article}[12pt]
\usepackage[utf8]{inputenc} 
\usepackage{amsthm}
\usepackage{hhline}
\usepackage{amssymb}
\usepackage{amsmath}
\usepackage{enumitem}
\usepackage{csquotes}
\usepackage{multirow}
\usepackage[
top    = 1.00cm,
bottom = 2.50cm,
left   = 3.00cm,
right  = 2.50cm]{geometry}

\newenvironment{solution}
  {\renewcommand\qedsymbol{$\blacksquare$}\begin{proof}[Solution]}
  {\end{proof}}
  
\newtheorem{innercustomprblm}{Problem}
\newenvironment{problem}[1]
  {\renewcommand\theinnercustomprblm{#1}\innercustomprblm}
  {\endinnercustomprblm}

\DeclareMathOperator{\Var}{Var}
\renewcommand{\P}{\mathbb{P}}
\newcommand{\E}{\mathbb{E}}

\title{Probability Theory \\ Homework Assignment 1}
\author{Bulat Khamitov, MSSA2019}
\date{September 22, 2019}

\begin{document}

\maketitle

\section{Axioms of Probability}

\begin{problem}{3}\normalfont
Two dice are thrown. 
Let E be the event that the sum of the dice is odd, let $F$ be the event that at least one of the dice lands on $1$, and let $G$ be the event that the sum is $5$. 
Describe the events $EF$, $E\cup F$, $FG$, $EF^{c}$, and $EFG$.
\end{problem}

\begin{solution}
We note that $EF = E \cap F$.
\begin{enumerate}
    \item $EF$ is the event that the sum of the dice is odd \textbf{and} at least one of the dice lands on $1$:
    \begin{equation*}
        EF = \left\{(1, 2),\, (1, 4),\, (1, 6),\, (2, 1),\, (4, 1),\, (6, 1)\right\}.
    \end{equation*}
    
    \item $E\cup F$ is the event that the sum of the dice is odd \textbf{or} at least one of the dice lands on $1$:
    \begin{equation*}
        E\cup F =
        \begin{Bmatrix}
        (1, 2), & (1, 4), & (1, 6), & (2, 1), & (4, 1), & (6, 1),
        \\
        (2, 3), & (2, 5), & (3, 2), & (3, 4), & (3, 6), & (4, 3),
        \\
        (4, 5), & (5, 2), & (5, 4), & (5, 6), & (6, 3), & (6, 5)
        \end{Bmatrix}.
    \end{equation*}
    
    \item $FG$ is the event that the sum of the dice is odd \textbf{and} the sum is $5$:
    \begin{equation*}
        FG = \left\{(1, 4),\, (4, 1)\right\}.
    \end{equation*}
    
    \item $EFG$ is the event that the sum is odd \textbf{and} the sum is $5$ \textbf{and} at least one of the dice shows a $1$:
    \begin{equation*}
        EFG = EG = \left\{(1, 4),\, (4, 1)\right\}.
    \end{equation*}
    
    \item $EF^{c}$ is the event that the sum of the dice is odd \textbf{and} at least one of the dice is \textbf{not} a $1$:
    \begin{equation*}
        EF^{c} = 
        \begin{Bmatrix}
        (2, 3), & (2, 5), & (3, 2), & (3, 4), & (3, 6), & (4, 3),
        \\
        (4, 5), & (5, 2), & (5, 4), & (5, 6), & (6, 3), & (6, 5)
        \end{Bmatrix}.
    \end{equation*}
\end{enumerate}
\end{solution}

\begin{problem}{7}\normalfont
Consider an experiment that consists of determining the type of job--either blue-collar or white-collar and the political affiliation--Republican, Democratic, or Independent—of the $15$ members of an adult soccer team.
How many outcomes are
\begin{enumerate}[label=(\alph*)]
    \item in the sample space?
    \item in the event that at least one of the team members is a blue-collar worker?
    \item in the event that none of the team members considers himself or herself an Independent?
\end{enumerate}
\end{problem}
\begin{solution}
 We denote $B$ and $W$ for blue-collar and white-collar respectively. 
 The political affiliations, Republican, Democratic or Independent are respectively denoted by $R$, $D$ and $I$.
\begin{enumerate}[label=(\alph*)]
    \item The sample space consists of all possible outcomes of this experiment.
    For each player there are $3\cdot 2 = 6$ possible outcomes, so the total number of outcomes is $|\Omega| = 6^{15}$.
    
    \item Let $F$ be the event that at least one of the team members is a blue-collar worker.
    It is easier to consider the event $F^{c}$, which is that all the team members are white-collar workers:
    \begin{equation*}
        F^{c} = \left\{\{W\}\times\{R, D, I\}^{15}\right\}.
    \end{equation*}
    Therefore,
    \begin{equation*}
        |F| = |\Omega| - |F^{c}| = 6^{15} - (1\cdot 3)^{15}.
    \end{equation*}
    
    \item Let $E$ be the event that none of the team members considers himself or herself an Independent. 
    If $E$ occurs, then all the members are either Republican or Democratic.
    \begin{equation*}
        E = \left\{\{B, W\}\times\{R, D\}^{15}\right\}.
    \end{equation*}
    Therefore,
    \begin{equation*}
        |E| = (2\cdot 2)^{15}.
    \end{equation*}
\end{enumerate}
\end{solution}

\begin{problem}{12}\normalfont
An elementary school is offering 3 language classes: one in Spanish, one in French, and one in German. 
The classes are open to any of the 100 students in the school. 
There are 28 students in the Spanish class, 26 in the French class, and 16 in the German class. 
There are 12 students that are in both Spanish and French, 4 that are in both Spanish and German, and 6 that are in both French and German. 
In addition, there are 2 students taking all 3 classes.
\begin{enumerate}[label=(\alph*)]
    \item If a student is chosen randomly, what is the probability that he or she is not in any of the language classes?
    \item If a student is chosen randomly, what is the probability that he or she is taking exactly one language class?
    \item If 2 students are chosen randomly, what is the probability that at least 1 is taking a language class?
\end{enumerate}
\end{problem}
\begin{solution}
We denote the events:
\begin{itemize}
    \item $S$ -- event that randomly chosen student studies Spanish;
    \item $F$ -- event that randomly chosen student studies French;
    \item $G$ -- event that randomly chosen student studies German.
\end{itemize}
\begin{enumerate}[label=(\alph*)]
    \item The condition of (a) implies that \enquote{\textbf{not} in Spanish \textbf{and not} in French \textbf{and not} in German}. Hence, we need to find 
    \begin{equation*}
        \P\left(S^{c}\cap F^{c}\cap G^{c}\right).
    \end{equation*}
    By De Morgan's law we have
    \begin{equation*}
         \P\left(S^{c}\cap F^{c}\cap G^{c}\right) = \P\left((S\cup F\cup G)^{c}\right) = 1 - \P\left(S\cup F\cup G\right).
    \end{equation*}
    By probability addition theorem we have
    \begin{align*}
        \P\left(S\cup F\cup G\right) &= \P(S) +  \P(F) +  \P(G) -
        \\
        & - \P(S\cap F) - \P(S\cap G) - \P(F\cap G) + \P(S\cap F\cap G).
    \end{align*}
    Using the conditions of the problem, we calculate the probabilities:
    \begin{align*}
        \P\left(S\cup F\cup G\right) &= \dfrac{28}{100} +  \dfrac{26}{100} +  \dfrac{16}{100} -
        \\
        & - \dfrac{12}{100} - \dfrac{4}{100} - \dfrac{6}{100} + \dfrac{2}{100} =
        \\
        & = \dfrac{50}{100}.
    \end{align*}
    Hence, 
    \begin{equation*}
        \P\left(S^{c}\cap F^{c}\cap G^{c}\right) = 1 - \P\left(S\cup F\cup G\right) = \dfrac{50}{100} = \dfrac{1}{2} = 0.5.
    \end{equation*}
    
    \item The condition of (b) implies that we need to find
    \begin{equation*}
        \P\left(\text{\enquote{exactly one}}\right). 
    \end{equation*}
    We know that there are 2 students taking all 3 classes, meaning:
    \begin{enumerate}[label=\arabic*)]
        \item \# of students in Spanish and French but not German: $12 - 2 = 10$;
        \item \# of students in both Spanish and German but not French: $4 - 2 = 2$;
        \item \# of students in both French and German but not Spanish: $6 - 2 = 4$.
    \end{enumerate}
    Hence, there are $10 + 4 + 2 = 16$ students taking 2 classes.
    From part (a) we know that 50 students take at least 1 class.
    So the number of students taking \textbf{exactly} 1 language class is $50 - 16 - 2 = 32$.
    Hence,
    \begin{equation*}
        \P\left(\text{\enquote{exactly one}}\right) = \dfrac{32}{100} = 0.32.
    \end{equation*}
    
    \item From part (a) we know  that 50 students take at least 1 class.
    We denote new events:
    \begin{itemize}
        \item $E_{1}$ -- student 1 takes a class;
        \item $E_{2}$ -- student 2 takes a class.
    \end{itemize}
    So we need to find
    \begin{align*}
        \P(E_{1}\cup E_{2}) &= \P(E_{1}) + \P(E_{1}) -\P(E_{1}\cap E_{2}) = 
        \\
        & = \dfrac{50}{100} + \dfrac{50}{100} - \dfrac{50}{100}\cdot\dfrac{49}{99} = \dfrac{149}{198} = 0.752525\ldots 
    \end{align*}
\end{enumerate}
\end{solution}

\begin{problem}{29a}\normalfont
An urn contains $n$ white and m black balls, where $n$ and $m$ are positive numbers.
\begin{enumerate}[label=(\alph*)]
    \item If two balls are randomly withdrawn, what is the probability that they are the same color?
\end{enumerate}
\end{problem}
\begin{solution}
The condition implies that we need to find
\begin{equation*}
    \P\left(\text{\enquote{same color}}\right) = \P\left(\text{\enquote{both white}}\right) + \P\left(\text{\enquote{both black}}\right).
\end{equation*}
When we withdraw $1$ ball out of $n + m$ balls in urn, there are $n + m - 1$ balls left.
And after doing so, there are either $n - 1$ white \textbf{or} $m - 1$ black balls left.
Hence,
\begin{equation*}
    \P\left(\text{\enquote{same color}}\right) = \binom{n}{n+m}\cdot\binom{n-1}{n+m-1} + \binom{m}{n+m}\cdot\binom{m-1}{n+m-1}.
\end{equation*}
After simplification we have
\begin{equation*}
    \P\left(\text{\enquote{same color}}\right) = \dfrac{n^{2} + m^{2} - n - m}{(n + m)(n + m -1)}.
\end{equation*}
\end{solution}

\begin{problem}{49}\normalfont
A group of $6$ men and $6$ women is randomly divided into $2$ groups of size $6$ each. 
What is the probability that both groups will have the same number of men?
\end{problem}
\begin{solution}
There are $\binom{6}{12}$ ways of splitting $12$ people into $2$ subgroups $6$ persons each.
Since we are interested in the same number of men in each group, there are $\binom{6}{3}$ ways for men and $\binom{6}{3}$ for women.
Hence,
\begin{equation*}
    \P\left(\text{\enquote{3 men \& 3 women in 1 group}}\right) = \dfrac{\binom{6}{3}\cdot\binom{6}{3}}{\binom{6}{12}} = 0.4329\ldots.
\end{equation*}

\end{solution}

\section{Conditional Probability and Independence}

\begin{problem}{3.19}\normalfont
A total of $48$ percent of the women and $37$ percent of the men that took a certain “quit smoking” class remained nonsmokers for at least one year after completing the class. 
These people then attended a success party at the end of a year. 
If 62 percent of the original class was male,
\begin{enumerate}[label=(\alph*)]
    \item what percentage of those attending the party were women?
    \item what percentage of the original class attended the party?
\end{enumerate}
\end{problem}
\begin{solution}
We consider the following events:
\begin{itemize}
    \item $S$ -- a randomly selected person attended a success party;
    \item $M$ -- a randomly selected man from the class;
    \item $F$ -- a randomly selected woman from the class;
\end{itemize}
Probabilities of being in the class:
\begin{equation*}
    \P(S\mid F) = 0.48,\quad\P(S\mid M) = 0.37,\quad\P(M) = 0.62.
\end{equation*}
\begin{enumerate}[label=(\alph*)]
    \item We need to find
    \begin{equation*}
        \P\left(\text{\enquote{women attended the party}}\right) = \P(F\mid S).
    \end{equation*}
    Since $62$ percent of the original class was male, then
    \begin{equation*}
        F = M^{c} \Longrightarrow\P(F) = 1 - \P(M) = 0.38.
    \end{equation*}
    By Bayes rule we have
    \begin{align*}
        \P(F\mid S) &= \dfrac{\P(S\mid F)\P(F)}{\P(S)} = 
        \\
        & = \dfrac{\P(S\mid F)\P(F)}{\P(S\mid F)\P(F) + \P(S\mid M)\P(M)} =
        \\
        & = \dfrac{0.48\cdot 0.38}{0.48\cdot 0.38 + 0.37 + 0.62} = 0.4429\ldots
    \end{align*}
    
    \item We need to find 
    \begin{equation*}
        \P\left(\text{\enquote{original class attended the party}}\right) = \P(S).
    \end{equation*}
    Having considered the results from part (a), we have
    \begin{align*}
        \P(S) &= \P(S\mid F)\P(F) + \P(S\mid M)\P(M) = 
        \\
        & = 0.48\cdot 0.38 + 0.37\cdot 0.62 = 0.4118
    \end{align*}
\end{enumerate}
\end{solution}

\begin{problem}{3.21}\normalfont
A total of 500 married working couples were polled about their annual salaries, with the following information resulting
\begin{table}[ph]
  \centering
  \begin{tabular}{c|c|c|c|c}
    \hline
    \multirow{2}{*}{Wife} & \multicolumn{2}{c}{Husband} \\
    \hhline{~--}
    & Less than \$ 25,000 & More than \$ 25,000 \\
    \hline
    Less than \$ 25,000 & 212 & 198 \\
    \hline
    More than \$ 25,000 & 36 & 54 \\
    \hline
  \end{tabular}
  \caption{Annual salaries}
  \label{T:peak}
\end{table}
For instance, in $36$ of the couples, the wife earned more and the husband earned less than \$$25,000$. 
If one of the couples is randomly chosen, what is
\begin{enumerate}[label=(\alph*)]
    \item the probability that the husband earns less than \$$25,000$?
    \item the conditional probability that the wife earns more than \$$25,000$ given that the husband earns more than this amount?
    \item the conditional probability that the wife earns more than \$$25,000$ given that the husband earns less than this amount?
\end{enumerate}
\end{problem}
\begin{solution}
We consider the following:
\begin{itemize}
    \item $HL$ -- \# of husbands earning less than \$$25,000$;
    \item $HM$ -- \# of husbands earning more than \$$25,000$;
    \item $WL$ -- \# of wives earning less than \$$25,000$;
    \item $WM$ -- \# of wives earning more than \$$25,000$.
\end{itemize}
Hence,
\begin{gather*}
    HL = 212 + 36 = 248
    \\
    HM = 198 + 54 = 252
    \\
    WL = 212 + 198 = 410
    \\
    WM = 36 + 54 = 90.
\end{gather*}
\begin{enumerate}[label=(\alph*)]
    \item Total number of husbands is
    \begin{equation*}
       H = HL + HM = 248 + 252 = 500.
    \end{equation*}
    So we need to find
    \begin{equation*}
        \P\left(\text{\enquote{husbands earn less}}\right) = \dfrac{HL}{H} = \dfrac{248}{500} = 0.496.
    \end{equation*}
    
    \item Number of wives earning more than \$25000 given that their husbands earn more han \$25000 is
    \begin{equation*}
        HM\cap WM = 54.
    \end{equation*}
    So we need to find
    \begin{equation*}
        \P\left(\text{\enquote{wives and husbands earn more}}\right) = \dfrac{HM\cap WM}{HM} = \dfrac{54}{252} = 0.2143.
    \end{equation*}
    
    \item Number of wives earning more than \$25000 given that their husbands earn less than \$25000 is
    \begin{equation*}
        HL\cap WM = 36.
    \end{equation*}
    So we need to find
    \begin{equation*}
        \P\left(\text{\enquote{wives earn more, husbands earn less}}\right) = 
        \dfrac{HL\cap WM}{HL} = \dfrac{36}{248} = 0.1452.
    \end{equation*}
\end{enumerate}
\end{solution}

\begin{problem}{3.26}\normalfont
Suppose that $5$ percent of men and $0.25$ percent of women are color blind. 
A color-blind person is chosen at random. 
What is the probability of this person being male? 
Assume that there are an equal number of males and females. 
What if the population consisted of twice as many males as females?
\end{problem}
\begin{solution}
We consider the following events:
\begin{itemize}
    \item $C$ -- a randomly selected person is color blind;
    \item $M$ -- a randomly selected person is man;
    \item $W$ -- a randomly selected person is woman.
\end{itemize}
Given probabilities are
\begin{equation*}
    \P(C\mid M) = 0.05,\quad \P(C\mid W) = 0.25.
\end{equation*}
\begin{enumerate}[label=(\roman*)]
    \item We assume that
    \begin{equation*}
        \P(M) = \P(W).
    \end{equation*}
    We need to find that
    \begin{equation*}
        \P(M\mid C).
    \end{equation*}
    By Bayes rule we have
    \begin{align*}
        \P(M\mid C) &= \dfrac{\P(C\mid M)\cdot\P(M)}{\P(C\mid M)\cdot\P(M) + \P(C\mid W)\cdot\P(W)} =
        \\
        & = \dfrac{0.05\cdot 0.5}{0.05\cdot 0.5 + 0.0025\cdot 0.5} = \dfrac{20}{21} = 0.9524\ldots
    \end{align*}
    
    \item We assume that
    \begin{equation*}
        \P(M) = 2\cdot\P(W).
    \end{equation*}
    Men and women are the whole population:
    \begin{equation*}
        \P(M) = 2\cdot\P(W) \Longrightarrow\P(M) = \dfrac{2}{3},\quad\P(F) = \dfrac{1}{3}.
    \end{equation*}
    Having considered Bayes rule we have
    \begin{equation*}
        \P(M\mid C) = \dfrac{0.05\cdot\frac{2}{3}}{0.05\cdot\frac{2}{3} + 0.0025\cdot\frac{1}{3}} = \dfrac{40}{41} = 0.9756\ldots
    \end{equation*}
\end{enumerate}
\end{solution}

\begin{problem}{3.57ab}\normalfont
A simplified model for the movement of the price of a stock supposes that on each day the stock's price either moves up $1$ unit with probability $p$ or moves down $1$ unit with probability $1 - p$. 
The changes on different days are assumed to be independent.
\begin{enumerate}[label=(\alph*)]
    \item What is the probability that after $2$ days the stock will be at its original price?
    \item What is the probability that after $3$ days the stock's price will have increased by $1$ unit?
\end{enumerate}
\end{problem}
\begin{solution}
We consider the following events:
\begin{itemize}
    \item $U_{i}$ -- the stock went up in the $i$-th day;
    \item $D_{i}$ -- the stock went down in the $i$-th day;
\end{itemize}
We note that $D_{i} = 1 - U_{i}$.
Given probabilities are
\begin{equation*}
    \P(U_{i}) = p,\quad \P(D_{i}) = 1 - p.
\end{equation*}
\begin{enumerate}[label=(\alph*)]
    \item We need to find 
    \begin{equation*}
        \P(U_{1}D_{2}\cup D_{1}U_{2}).
    \end{equation*}
    Since the events are mutually exclusive, we have
    \begin{align*}
        \P(U_{1}D_{2}\cup D_{1}U_{2}) &= \P(U_{1}D_{2}) +\P(D_{1}U_{2}) = 
        \\
        & = \P(U_{1})\P(D_{2}) +\P(D_{1})\P(U_{2}) = 
        \\
        &= p(1 - p) + (1 - p)p
        \\
        &= 2p(1 - p),
    \end{align*}
    where $U_{1}$ and $U_{2}$ are independent.
    
    \item We need to find
    \begin{equation*}
        \P(U_{1}D_{2}U_{3}\cup D_{1}U_{2}U_{3}\cup U_{1}U_{2}D_{3}).
    \end{equation*}
    Since the events are mutually exclusive, we have
    \begin{align*}
        \P(U_{1}D_{2}U_{3}\cup D_{1}U_{2}U_{3}\cup U_{1}U_{2}D_{3}) &=  \P(U_{1}D_{2}U_{3}) +\P(D_{1}U_{2}U_{3}) +\P(U_{1}U_{2}D_{3}) =
        \\
        &= \P(U_{1})\P(D_{2})\P(U_{3}) + 
        \\
        &+ \P(D_{1})\P(U_{2})\P(U_{3}) + 
        \\
        &+ \P(U_{1})\P(U_{2})\P(D_{3}) =
        \\
        &= p^{2}(1 - p) + (1 - p)p^{2} + p(1 - p)p =
        \\
        & = 3p^{2}(1 - p).
    \end{align*}
    \end{enumerate}
\end{solution}

\begin{problem}{3.60a}\normalfont
The color of a person’s eyes is determined by a single pair of genes. 
If they are both blue-eyed genes, then the person will have blue eyes; if they are both brown-eyed genes, then the person will have brown eyes; and if one of them is a blue-eyed gene and the other a brown-eyed gene, then the person will have brown eyes. 
(Because of the latter fact, we say that the brown-eyed gene is \textit{dominant} over the blue-eyed one.) 
A newborn child independently receives one eye gene from each of its parents, and the gene it receives from a parent is equally likely to be either of the two eye genes of that parent. 
Suppose that Smith and both of his parents have brown eyes, but Smith’s sister has blue eyes.
\begin{enumerate}[label=(\alph*)]
    \item What is the probability that Smith possesses a blue-eyed gene?
\end{enumerate}
\end{problem}
\begin{solution}
We consider the following events:
\begin{itemize}
    \item $Br$ -- dominant gene for brown eyes;
    \item $Bl$ -- non-dominant gene for blue eyes;
\end{itemize}
Possible outcomes:
\begin{equation*}
    \left\{(Br, Br),\, (Br, Bl),\, (Bl, Br),\, (Bl, Bl)\right\}.
\end{equation*}
For having brown eyes:
\begin{equation*}
    \left\{(Br, Br),\, (Br, Bl),\, (Bl, Br)\right\}.
\end{equation*}
\begin{enumerate}[label=(\alph*)]
    \item Satisfying events are
    \begin{equation*}
         \P\left(\text{Smith $= (Bl, Br)$ $\cup$ Smith $= (Br, Bl)$}\right).
    \end{equation*}
    Since the events are mutually exclusive and equally probable, we have
    \begin{multline*}
        \P\left(\text{Smith $= (Bl, Br)$ $\cup$ Smith $= (Br, Bl)$}\right) =
        \\
        = \P\left(\text{Smith $= (Bl, Br)$}\right) + \P\left(\text{Smith $= (Br, Bl)$}\right) = \dfrac{1}{3} + \dfrac{1}{3} = \dfrac{2}{3}.
    \end{multline*}
\end{enumerate}
\end{solution}

\begin{thebibliography}{00}
\bibitem{b1} Sheldon Ross. \textit{A first course in probability}. Eighth Edition. Pearson
Education, Inc. 2010
\end{thebibliography}

\end{document}